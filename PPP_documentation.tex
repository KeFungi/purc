%% Based on the GSI document, so mostly from Swofford's pre-existing styles

%% Remember to compile as XeLaTeX (rather than just LaTeX)

% Simple command-line command to get a word count breakdown: texcount mytexfile.tex

% quick command refs:
%\textrm   \rmfamily   Roman family
%\textsf   \sffamily   Sans serif family
%\texttt   \ttfamily   Typewriter family
%\textup   \upshape    Upright shape
%\textit   \itshape    Italic shape
%\textsl   \slshape    Slanted shape
%\textsc   \scshape    Small caps shape
%\textmd   \mdseries   Medium series
%\textbf   \bfseries   Boldface series

\documentclass[12pt,letterpaper]{article}

\usepackage{fixltx2e}
\usepackage{textcomp}
\usepackage{fullpage}
\usepackage{amsfonts}
\usepackage{verbatim}
\usepackage[english]{babel}
\usepackage{pifont}
\usepackage{color}
\usepackage{setspace}
\usepackage{textcase}
\usepackage{lscape}
\usepackage{indentfirst}
\usepackage[normalem]{ulem}
\usepackage{booktabs}
%\usepackage{nag}
\usepackage{xspace}

\usepackage{natbib}
\bibliographystyle{evolution}
\bibpunct{(}{)}{;}{a}{}{,}  % this is a citation format command for natbib

\usepackage{float}
\usepackage{latexsym}
\usepackage[colorlinks=true,linkcolor=blue]{hyperref} % I think this allows for hyperlinks within the document. E.g., in the table of contents to the respecitive parts of the document
\usepackage{url}
%\usepackage{html}
\usepackage{epsfig}
\usepackage{graphicx}
\usepackage{amssymb}
\usepackage{amsmath}
\usepackage{bm}
\usepackage{array}
\usepackage[version=3]{mhchem} % Without the version= this threw up an error when I typeset
\usepackage{ifthen}
\usepackage{caption}
%\usepackage{xcolor}
\usepackage{amsthm}
\usepackage{amstext}

\usepackage{newicktree} % only if we're going to be drawing some newick trees

\usepackage[big,sf,bf]{titlesec} % from Dave's PAUP manual -- trying to get the sections to show up in the TOC

%DLS: paralist is a package for some alternative list environments
\usepackage{paralist}

% DLS: I use the todonotes package to mark up text with 'to do' items.  I add the 'todoin'
%      command for an inline todo with a different appearance.
\usepackage{todonotes}
\newcommand{\todoin}[1]{
  {\singlespacing\todo[inline, size=\small, caption={2do},backgroundcolor=blue!5!white, bordercolor=blue]{#1}}
}

%DLS: My standard font stuff...
\ifxetex
	\usepackage{xltxtra,xunicode}
	\defaultfontfeatures{Mapping=tex-text}
	\setromanfont[Mapping=tex-text]{Palatino}
	\setsansfont[Scale=MatchLowercase,Mapping=tex-text]{Gill Sans}
	\setmonofont[Scale=MatchLowercase]{Andale Mono}
	\usepackage{unicode-math}
%	\setmathfont{xits-math.otf}
	\setmathfont{Asana-Math.otf}
	\newfontfamily\applekeyfont{Lucida Grande}
	\newcommand\cmdkey[1]{{\applekeyfont \char"2318 #1}}
	\newcommand\optionkey[1]{{\applekeyfont \char"2325 #1}}
	\newcommand\shiftkey[1]{{\applekeyfont \char"21E7 #1}}
	\newcommand\tabkey[0]{{\applekeyfont \char"21E5}}
	\newcommand\deletekey[0]{{\applekeyfont \char"232B}}
\else
	\usepackage[utf8]{inputenc}
	\usepackage[T1]{fontenc}
\fi

% Heading formats, from Dave's PAUP manual
%\titleformat{\chapter}{\Huge\sffamily\bfseries\hfill }{Chapter \fontsize{84}{84} \selectfont \thechapter }{0em}{\\}
\titleformat{\chapter}
	{\Huge\sffamily\bfseries\raggedleft}
	{\fontsize{84}{84} \selectfont \thechapter}  %{\fontsize{84}{84} \selectfont \color{hilite} \thechapter}
	{0em}
%	{\\ \color{hilite}}
        {}
\titleformat{\section}
	{\LARGE\sffamily\itshape}  %{\LARGE\sffamily\itshape \color{hilite}}
	{\thesection}
	{0.5em}
	{}
\titleformat{\subsection}
	{\Large\itshape}
	{\thesubsection}
	{0.5em}
	{}
\titleformat{\subsubsection}
	{\Small\sffamily}
	{\thesubsubsection}
	{0.5em}
	{}
	
\newcommand{\figref}[1]{Fig. \ref{#1}}

\linespread{1.66}
\raggedright
\setlength{\parindent}{0.5in}
%\setcounter{secnumdepth}{0} % This command will suppress the section numbers

\pagestyle{empty}

%-------------- Formatting Table of Contents --------------
\setcounter{tocdepth}{5} % controls how "deep" the table of contents goes




%%%%%%%%%%% Title Page Stuff

\begin{document}
\begin{flushright}
Version dated: \today
\end{flushright}
\bigskip

\noindent RH: PPP or PURR or PUSS or PUS or..:
\bigskip
\medskip

\begin{center}
\noindent{\Large \bf PPP/PUSS/Whatever v0.1 Manual}\\
\bigskip
\noindent {\normalsize \bf Carl J. Rothfels$^1$, and Fay-Wei Li$^{1,2}$}\\
\noindent {\small \it
$^1$University Herbarium and Department of Integrative Biology, University of California, Berkeley, California, 94720-2465\\
$^2$Department of Biology, Duke University, Durham, NC 27708}\\
\end{center}
\medskip

\noindent{\bf Corresponding author:} Carl J. Rothfels, University of California, Berkeley, California, USA; E-mail: crothfels@yahoo.ca\\
\vspace{0.5in}


\vspace{0.5in}
\noindent \textbf{Keywords:} polyploidy, next-generation sequencing, PacBio, Pacific BioSciences, reticulate evolution, multi-labeled gene trees, 



%%%%%%%%%%% TOC
\tableofcontents
%\listoffigures
%\listoftables


%%%%%%%%%%% Main document
\newpage
\section{TODO list}
\todoin{
\begin{compactitem}
\item Check into usearch 8  \textbf{DONE! \textit{ha.}}
\item Reformatting reference sequences and maps  \textbf{DONE! \textit{ha.}}
\item Carl will continue to look at the data
\item Have it export the alignments in nexus format
\item Carl to create rough outline of documentation file. \textbf{DONE! \textit{ha.}}
\end{compactitem}
}

\bigskip\section{Introduction}
\subsection{Background} 
Discussion of the issues with homeologs and paralogs, and the utility of 
multilabeled genetrees. Issues with cloning. Some terminolog, etc.
Quick intro to what PPP (to be renamed) does.

\subsection{PPP workflow}
Make a flowchart? Describe the things that PPP does.
Describe how to call it.

\bigskip\section{Dependencies} %Describe how to install all of these
\subsection{Python}
Requires Python 2.7 or later version. We have not tested PPP on Python 3. 

\subsection{BioPython}
Requires BioPython 1.6 or later version (http://biopython.org/wiki/Main_Page). You have to have Numpy (http://www.numpy.org) in place before installing BioPython. 

\subsection{Blast}
BLAST+ (http://blast.ncbi.nlm.nih.gov/Blast.cgi?PAGE_TYPE=BlastDocs&DOC_TYPE=Download) is required, and should be in your PATH. If you are using Mac, the easiest way is to download the .dmg file and follow the installer's instruction.  

\subsection{USEARCH}
Requires Usearch 8 (http://www.drive5.com/usearch/), and PPP is not compatible with Usearch 7.

\subsection{MUSCLE}
http://www.drive5.com/muscle/

\subsection{etc}

\bigskip\section{Input data}
Description of how to generate the sorts of data that PPP can use.

\bigskip\section{Supporting files}
\subsection{Configuration file}
The configuration file is a text file... blah blah, where the user can
set the following options:

\subsubsection{option1} % This seems to work, although the compiler complains
There are four options for \textbf{mode} ... 

\subsubsection{option2}

\subsubsection{option3}

\subsection{Map files}

\subsection{Primer file}

\subsection{Barcode file}

\subsection{Reference sequences}

\bigskip\section{Output files}
Describe each of the outputs, and what they contain.

\subsection{unmatched.fa}

\bigskip\section{Troubleshooting}
\subsection{Common errors}
\subsection{Error messages}


%%%%%%%%% Acknowledgments
\bigskip\section{Acknowledgements}
We thank .... NESCent short-term scholar support.. 

%
%Tree diagrams were
%produced with the \textsf{newicktree} \LaTeX package \citep{Savva:2004}.
%
%\newpage
%
%Example from \citet{Savva:2004}:
%\begin{newicktree}
%\drawtree{((My:1,first:1.5):0.5,(\sf newicktree:2,tree!:2.5):0.5):0.5;}
%\end{newicktree}
%
%
%
%\pagebreak
%
%---------------------------------bibliography------------------------------------


\bibliography{PPP_documentation}

%------------------------------------figures------------------------------------
%\pagebreak

%\begin{figure} %[h]
%\includegraphics[width=0.9\textwidth]{figures/fig_WithOutgroup.pdf}
%%\missingfigure[figwidth=6cm]{Sampling \ldots}
%\caption{Effect of sampling regime on ....}
%\label{fig:w_outgroup}
%\end{figure}

\end{document}
